\documentclass[fleqn]{homework}

\student{Stephen Brennan (smb196)}
\course{EECS 477}
\assignment{Homework 1}
\duedate{September 11, 2015}

\usepackage{amsmath}
\usepackage{amsfonts}
%\usepackage{graphicx}

\begin{document}
  \maketitle

  \begin{problem}{1}
    \begin{question}
      Consider a greedy algorithm for Vertex Cover in which we repeatedly choose
      a vertex of maximum degree, add it to the vertex cover, and then remove it
      and all its incident edges. \\
      \textbf{(a)} Show that this algorithm is optimal on the two graphs
      discussed in class: \textbf{(i)} an $n$ vertex star graph, \textbf{(ii)}
      an $K_4$ minus one edge (where $K_n$ is the complete graph on $n$ vertices).\\
      \textbf{(b)} Show that this algorithm is not a 2-approximation algorithm.
    \end{question}
  \end{problem}

  \begin{problem}{2}
    \begin{question}
      Consider the following linear program:
      \begin{align*}
        \max 3x_1 - 5x_2 & \\
        \text{s.t.  } 4x_1 + 5x_2 &\ge 3 \\
        6(x_1 - x_2) &= 7 \\
        x_1 + 8x_2 &\le 20 \\
        x_1, x_2 &\ge 0
      \end{align*}
      \textbf{(a)} Put this program in standard form.\\
      \textbf{(b)} Use octave to solve the original linear program.\\
      \textit{Postponed to next assignment: \textbf{(c)} State the corresponding
        optimal basic solutions in terms of the matrices $\mathcal{B}$ and
        $\mathcal{L}$, and state the corresponding values of $\pi$.}
    \end{question}
  \end{problem}

  \begin{problem}{3}
    \begin{question}
      Use Octave to solve the following integer linear program:
      \begin{align*}
        \max 3x_1 - 5x_2 & \\
        \text{s.t.  } 4x_1 + 5x_2 &\ge 3 \\
        6(x_1 - x_2) &= 7 \\
        x_1 + 8x_2 &\le 20 \\
        x_1, x_2 &\text{ integer}
      \end{align*}
    \end{question}
  \end{problem}

  \begin{problem}{4}
    \begin{question}
      In this problem, you will consider an instance of the fractional knapsack
      problem in which there are five items of weights 5, 10, 20, 30, 40 and
      benefit 30, 20, 100, 90, 160 (the benefit is the profit gained by loading
      the full item in the knapsack).  Formulate the problem as a linear program
      and use octave to solve it.
    \end{question}
  \end{problem}

  \begin{problem}{5}
    \begin{question}
      Suppose that you have the same items as in the previous problem, but you
      wish to solve an integer knapsack problem.  Formulate the problem as an
      integer linear program and use octave to solve it.  If the solution
      differs from that of the previous problem, write a short intuitive
      explanation to justify the discrepancy.
    \end{question}
  \end{problem}

  \begin{problem}{6}
    \begin{question}
      In this question, you are asked to create an instance of the knapsack
      problem in which there is a large discrepancy between the fractional and
      integral version.  Specifically, let

      \begin{align*}
        F &= \max \{c^T x: w^T x \leq W, 0 \leq x \leq e\}\\
        I &= \max \{c^T x: w^T x \leq W, x\in\{0,1\}^n\}
      \end{align*}

      where $x$ is an $n$ vector and $e$ is the $n$ vector consisting of 1s.  In
      this problem, you will attempt to make the ratio $F/I$ as large as
      possible while $I>0$.  Specifically, either give an octave example with
      appropriately chosen values of $c$, $w$, and $W$, or prove that $F/I$
      (when $I>0$) can be made arbitrarily large.
    \end{question}
  \end{problem}

  \begin{problem}{7}
    \begin{question}
      A corporation outsources work to external contractors.  The work is done
      by teams, each of which is capable of doing all types of work.  The work
      is divided across several geographic districts, where the $j$th district
      requires exactly $r_j$ teams.  The contractors are of two types:
      experienced and inexperienced.

      Contractor $i$ can supply at most $u_i$ teams and quotes a price $c_{ij}$
      to have one team conduct the work in the $j$th district.  The objective is
      to allocate the work in the districts to various contractors so that each
      district has at least one experienced contractor assigned to it.  The
      values of $r_j$ and $u_i$ are integers.\\
      \textbf{(a)} Formulate the problem as an integer linear program.\\
      \textbf{(b)} Formulate its linear relaxation.\\
      \textbf{(c)} Give an example with appropriately chosen values of $r_i$,
    $u_i$, and $c_{ij}$ where there is an optimal basic solution that is not
    integer, or document at least five attempts to find such an example.
    \end{question}
  \end{problem}

\end{document}