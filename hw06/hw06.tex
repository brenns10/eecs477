\documentclass[fleqn]{homework}

\student{Stephen Brennan (smb196)}
\course{EECS 477}
\assignment{Homework 6}
\duedate{December 4, 2015}

\usepackage{algpseudocode}
\usepackage{enumerate}
\usepackage{mathtools}
%\usepackage{graphicx}

\begin{document}
  \maketitle

  \begin{problem}{1}
    \begin{question}
      A manufacturer of pharmaceutical substances has a product line consisting
      of $n$ substances. The substances have completely different properties,
      yet two substances can be distinguished only through sophisticated
      chemical tests. There are $k$ possible chemical tests and, in each test, a
      substance either tests positive or tests negative. The manufacturer wishes
      to assign a product code to each substance. The product code is a binary
      string representing the test outcome for that particular substance. For
      example, if $k=4$, and a substance passes test 1, 3, and 4 but fails test
      2, its code would be 1011.  However, some of the tests are redundant: for
      example, there is no point to encode a test that is passed by all of the
      $n$ available pharmaceutical substances, nor a test whose outcome could be
      inferred by that of another test(s). The manufacturer would like to
      determine if $t<k$ tests are sufficient for the encoding. In other words,
      he needs $t$ out of the $k$ available tests with the property that any two
      of the $n$ substances differ on at least one of the $t$ test outcomes.
      \begin{enumerate}[a.]
      \item Argue that the manufacturer needs at least $t=\Omega(\log n)$ tests
        to differentiate the $n$ substances from each other.
      \item Model the manufacturer's problem as an integer linear program.
      \item Use the program to devise a pimal-dual approximation algorithm for
        the manufacturer's problem.
      \item Use the program to devise a randomized approximation algorithm for
        the manufacturer's problem.
      \end{enumerate}
    \end{question}
    \begin{enumerate}[a.]
    \item For $t$ tests, there are $2^t$ product codes, so there can be at most
      $2^t$ products distinguished by these tests (and this is only if the tests
      provide the best case amount of information).  It follows that for $n$
      products, the minimum number of tests is $\lceil \log n \rceil$, so
      $t \in \Omega(\log n)$.
    \item We define the variable $x_i$ for $1 \le i \le k$ as follows:
      \begin{equation*}
        x_i = \left\{\begin{array}{ll}
            1 &\text{ if test }i\text{ is included in the set} \\
            0 &\text{ otherwise}
          \end{array}\right.
      \end{equation*}
      These $x_i$ are the variables we need to find.  Similarly, we define the
      variables $d_{iab}$ for $a,b \in [1, n]$:
      \begin{equation*}
        d_{iab} = \left\{\begin{array}{ll}
                           1 &\text{ if test }i\text{ distinguishes substances }a,b \\
                           0 &\text{ otherwise}
                         \end{array}\right.
      \end{equation*}
      These $d_{iab}$ variables are inputs to the problem.  With this
      formulation, we can define the following integer linear program to solve
      this problem:

      \begin{align*}
        \min \sum_{i=1}^k x_i &\text{ s.t.} \\
        \sum_{i=1}^k d_{iab} x_i \ge 0 &\:\:\:\: \forall a,b \in [1,n]: a \ne b \\
        x_i \in \{0,1\} &\:\:\:\: \forall i \in [1, k] \\
      \end{align*}
    \item Here is the beginnings of the formulation of the dual of this program:
      \begin{align*}
        \max 0 &\text{ s.t.} \\
        \sum_{a,b \in [1,n]: a \ne b} d_{iab} \alpha_{ab} \le 1 &\:\:\:\: \forall i \in [1,k] \\
      \end{align*}

      While this dual is not complete, it can give us the following
      complementary slackness conditions:

      \begin{enumerate}
      \item $x_i \left(\sum_{a,b \in [1,n]: a \ne b} d_{iab} \alpha_{ab} - 1\right) = 0$
      \item $\alpha_{ab} \left(\sum_{i=0}^k x_i d_{iab}\right) = 0$
      \end{enumerate}

      We can use this to come up with a primal-dual algorithm very similar to
      the one presented in class for Hitting Set:

      \begin{algorithmic}
        \Function{Primal-Dual}{$d_{ab}$ values}
          \State $\vec{x} \gets 0$
          \State $\vec{\alpha} \gets 0$
          \While{$\exists a,b: \sum_{i=1}^k x_i d_{iab} = 0$}
            \Comment{while $\vec{x}$ is still infeasible}
            \State increase $\alpha_{ab}$ until $\exists x_i: \sum_{a,b} \alpha_{ab} d_{iab} - 1 = 0$.
            \State $x_i \gets 1$
          \EndWhile
          \State \Return $\vec{x}$
        \EndFunction
      \end{algorithmic}
      This algorithm starts with an $\vec{x}$ and $\vec{\alpha}$ linked by
      complementary slackness, but $\vec{x} = 0$ is not feasible (it represents
      an empty set of tests).  At each iteration, it selects a pair of
      substances which cannot be distinguished with the current set of tests.
      It increases the $\alpha$ value until a test is identified that will
      distinguish them.  This continues until all pairs of substances can be
      distinguished, at which point the set of $t$ tests are the tests $i$ such
      that $x_i=1$.
    \item The linear relaxation of the primal ILP is:
      \begin{align*}
        \min \sum_{i=1}^k x_i &\text{ s.t.} \\
        \sum_{i=1}^k d_{iab} x_i \ge 0 &\:\:\:\: \forall a,b \in [1,n]: a \ne b \\
        0 \le x_i \le 1 &\:\:\:\: \forall i \in [1, k] \\
      \end{align*}
      Thus, the randomized approximation algorithm associated with the
      manufacture's problem would be to solve this linear program, treat the
      $x_i$'s as probabilities, and include each test $i$ in the set of tests
      with probability $x_i$.  This would be repeated multiple times so as to
      increase the probability that the solution is feasible.
    \end{enumerate}

  \end{problem}

  \begin{problem}{2}
    \begin{question}
      The \textit{Bloom filter} is a data structure that is used to represent
      sets in applications that require extreme speed at the potential of a
      small error probability (e.g., packet logging at line speed in
      networks). The Bloom filter uses $k$ hash functions $h_1, h_2, ..., h_k$
      and a binary array $A$ of size $m$. You can assume that $h_i(x)$ is
      equally likely to be any index in the array $A$. Initially, the array $A$
      is 0. An element $x$ is added to set by setting
      $A[h_1(x)], A[h_2(x)], \dots, A[h_k(x)]$ to 1. To check whether $x$
      belongs to the set, the data structure tests whether
      $A[h_1(x)], A[h_2(x)], \dots, A[h_k(x)]$ are all equal to 1: if so, $x$ is
      reported as being in the set, otherwise $x$ is reported as not being in
      the set. Due to potential collisions in the hashing functions, it is
      possible that $x$ is reported as belonging to the set even if it does
      not. However, the data structure is always correct when it reports that
      $x$ does not belong to the set. The data structure does not support
      removal of elements from the set. Assuming that the $n<km/2$ elements have
      been inserted in the Bloom filter, derive a bound on the probability that
      the Bloom filter returns a false positive response.
    \end{question}

    Since the hash function values are equally likely, the probability that the
    hash of any element is equal to any array index is $\frac{1}{m}$.  We can
    use this to compute the probability that $A[j] = 1$:

    \begin{align*}
      Pr[A[j] = 1] &= 1 - Pr[A[j] = 0] \\
                   &= 1 - Pr\left[\bigcap_{i=1}^n \bigcap_{\ell=1}^k \{h_\ell(x_i) \ne j\}\right] \\
                   &= 1 - \prod_{i=1}^n \prod_{\ell=1}^k \left(1 - \frac{1}{m}\right) \\
                   &= 1 - \left(1 - \frac{1}{m}\right)^{nk}
    \end{align*}

    From this, we can determine the probability that $k$ indices are all 1:

    \begin{align*}
      Pr[A[h_\ell(x)] = 1, 1 \le \ell \le k] &= \left(1 - \left(1 - \frac{1}{m}\right)^{nk}\right)^k
    \end{align*}

    However, in the worst-case for the filter, all $k$ hash functions evaluate
    to the same index, and so the worst case probability of a false positive is
    simply:

    \begin{align*}
      Pr[\text{false positive}] \le 1 - \left(1 - \frac{1}{m}\right)^{nk}
    \end{align*}

    We might try to simplify this upper bound by noting that since
    $n < \frac{mk}{2}$, we also have $nk < \frac{mk^2}{2}$.  This would mean
    that:

    \begin{align*}
      \left(1 - \frac{1}{m}\right)^{nk} \ge 
      \left(1 - \frac{1}{m}\right)^{\frac{mk^2}{2}} = 
      \left(\left(1 - \frac{1}{m}\right)^m\right)^{\frac{k^2}{2}}
    \end{align*}

    However, from there, we can only say that
    $\left(1 - \frac{1}{m}\right)^m \le e^{-1}$, and this will not let us say
    anything meaningful about the original bound.  So, we are stuck with the
    bound on the false positive rate given above.
  \end{problem}

  \begin{problem}{3}
    \begin{question}
      The \textit{quadratic relaxation} of MAX CUT is:
      \begin{equation*}
        \max \left\{\sum a_{ij} \frac{1 - v_i \cdot v_j}{2}: -e \le v_i \le e\right\},
      \end{equation*}
      where the $v_i$'s are $n$-dimensional vectors, the underlying graph has
      $n$ vertices, and $e$ is the vector of all ones.
      \begin{enumerate}[a.]
      \item Show that, in the optimal solution of this optimization problem,
        every non-zero vector has (at least) a component of value $+1$ or $-1$.
      \item Show that there exists an optimal solution where no vector is zero.
      \item Briefly summarize the main similarities and differences of the
        quadratic and the semi-definite programming relaxations of MAX CUT.
      \item In the next question, you will be using the following general fact
        on vectors and matrixes. Let $v_1$ and $v_2$ be two $n$-dimensional
        vectors and define the $2n$-dimensional vector
        \begin{equation*}
          v = \begin{bmatrix*} v_1 \\ v_2 \end{bmatrix*}
        \end{equation*}
        Define a matrix $H$ such that $v_1 v_2 = v^T H v$.
      \item Write the MAX CUT quadratic relaxation as a \textit{quadratic
          program} of the form $\min\{x^TQx: -e \le x \le e \}$, where $e$ is
        the vector of all ones.
      \end{enumerate}
    \end{question}

    \begin{enumerate}[a.]
    \item Assume that there exists an optimal solution for this problem, where
      there is a vector $v_i$ which is nonzero but none of its components are
      $\pm 1$.  Let one such nonzero component be indexed by $k$.  It must be
      the case that $\sum_{j} -v_i(k) * v_j(k)$ is positive (otherwise,
      $v_{i}(k)$ would be zero, thus improving the objective function).  So, by
      increasing the magnitude of $v_i(k)$ until it reaches $\pm 1$, we can
      create a new solution which is also feasible, and has a better objective
      value.  This gives a contradiction.  Therefore, if a solution is optimal,
      then all nonzero vectors must have at least one component that is $\pm 1$.
    \item Assume there is an optimal solution which has a zero vector $v_i$.
      Take any element $k$ of that vector, and look at the component that
      contributes to the objective function:
      $\sum_{j} - v_i(k) * v_j(k) = - v_i(k) \sum_{j} v_j(k)$.  If
      $\sum_{j} v_j(k)$ is positive or negative, then the solution must not have
      been optimal to begin with (you can improve the objective value by setting
      $v_i(k)$.  Therefore, $\sum_{j} v_j(k) = 0$.  This means that you can set
      $v_i(k)$ to 1 or -1, obtaining an equivalent optimal feasible solution
      that has no nonzero vectors.
    \item The quadratic programming relaxation has a hypercube as the feasible
      region, because the constraints are linear in the vectors.  The SDP as a
      sphere as the feasible region, because the constraints are linear in the
      dot products of the vectors.
    \item Below:
      \begin{align*}
        H = \begin{bmatrix*} 
          0 & I \\
          0 & 0 \\
        \end{bmatrix*}
      \end{align*}
      Each entry in the above matrix is a square matrix of size $n$ by $n$.  0
      is the $n$ by $n$ zero matrix, and $I$ is the $n$ by $n$ identity.
    \item Let
      \begin{align*}
        x = \begin{bmatrix*} v_1 \\ v_1 \\ v_2 \\ v_2 \\ \vdots \\ v_n \\ v_n \end{bmatrix*}
      \end{align*}

      and

      \begin{align*}
        Q = \begin{bmatrix*} 
          0 & 0 & \dots & 0 & 0 \\
          a_{21} H & 0 & \dots & 0 & 0 \\
          a_{31} H & a_{32} H & \dots & 0 & 0 \\
          \vdots & \vdots & \ddots & \vdots & \vdots \\
          a_{n1} H & a_{n2} H & \dots & a_{n,n-1}H & 0 \\
        \end{bmatrix*}
      \end{align*}

      $x$ is a $2n^2$ length vector, and $Q$ is $2n^2$ by $2n^2$ matrix.  Each
      entry in $Q$ above is either the $2n$ by $2n$ zero matrix, or the $2n$ by
      $2n$ matrix $H$ defined in problem d.  When you take the product $x^T Q x$
      you obtain the original formula.
    \end{enumerate}
  \end{problem}

  \begin{problem}{4}
    \begin{question}
      Implement and solve the MAX CUT quadratic program in octave for a complete
      graph with three vertices, with weights $a_{ij} > 0$ of your choice.
    \end{question}
  \end{problem}

  \begin{problem}{5}
    \begin{question}
      A sequence of n operations is performed on a data structure.  The $i$th
      operation takes time $i$ if $i$ is a power of 2, and 1
      otherwise. Determine the amortized cost per operation using
      \begin{enumerate}[a.]
      \item Aggregate analysis
      \item The accounting method
      \item A potential function
      \end{enumerate}
    \end{question}

    \begin{enumerate}[a.]
    \item For simplicity, assume $n$ itself is a power of 2.  Then, there are
      $\log n + 1$ operations $i$ which are powers of 2, and $n - \log n - 1$
      operations which are not.  So, the total cost of the operations that are
      not powers of two is $n - \log n - 1$.  The total cost of the operations
      that are powers of 2 is equal to
      $\sum_{i=0}^{\log n} 2^i = 2^{\log n + 1} - 1$.  Therefore, the overall
      total cost of the operations on the data structure is:

      \begin{align*}
        \text{total cost} = n - \log n - 1 + 2n - 1 = 3n - \log n - 2
      \end{align*}

      Therefore the amortized cost of each operation (when there are a total of
      $n$ operations, and $n$ is a power of 2) is:

      \begin{align*}
        \text{amortized cost} = 3 - \frac{\log n}{n} - \frac{2}{n}
      \end{align*}

      Note that as $n$ grows, the amortized cost approaches 3.  Furthermore,
      when $n$ is not a power of two, the total cost is lower than what is
      calculated above.  Therefore, we can always say that the amortized cost of
      any operation on this data structure is less than or equal to 3.
    \item In the accounting method, each operation $i$ will have cost 1, and
      will ``account'' for a share of the cost of the very next power of 2.
      More explicitly:
      \begin{itemize}
      \item When $i$ is not a power of 2, the operation will have a cost of 1,
        and the operation will ``leave a candy'' of value $2 - \frac{2}{x}$,
        where $x = \lceil \log i \rceil$.  Its total cost is $3 - \frac{2}{x}$.
      \item When $i$ is a power of 2, the operation will have cost $i$, but it
        will be able to ``eat'' the candy left by the operations between the
        previous power of 2 and the current one.  There are $\frac{i}{2} - 1$
        operations between, and each one left $2 - \frac{2}{i}$ candy, so this
        operation eats
        $\left(\frac{i}{2} - 1\right)\left(2 - \frac{2}{i}\right) = i - 3 +
        \frac{2}{i}$
        candy.  This leaves a total cost of this operation that is
        $i - i + 3 - \frac{2}{i} = 3 - \frac{2}{i}$.
      \end{itemize}

      Again we see that both types of operations have amortized cost bounded by
      3, just as we saw above.
    \item In the potential method, we will have pretty much the exact same
      formulation as the accounting method.  The potential function is
      $\Phi(D_i) =$ the amount of candy saved up after operation $i$.  So,
      $\Phi(D_i) = 0$ for each $i$ which is a power of 2, and
      $\Phi(D_i) = \Phi(D_{i-1}) + 2 - \frac{2}{\lceil \log i \rceil}$ for $i$
      which are not.  Therefore, the amortized cost of each operation $i$ is
      simply $\hat{c}_i = 3 - \frac{2}{\lceil \log i\rceil}$ (as we saw
      previously), and this is bounded by 3, which is again what we've seen all
      along.
    \end{enumerate}
  \end{problem}

  \begin{problem}{6}
    \begin{question}
      Consider the \textit{Least Frequently Used (LFU)} caching rule, where the
      cached page that has been accessed the least often is the one that is
      evicted when the cache is full and a new page is requested. If there are
      ties, the page that has been in the page the longest is evicted. Show a
      sequence of $n$ requests that forces LFU to miss $\Omega(n)$ times for a
      cache of size $m$, whereas the optimal algorithm will only miss $O(m)$
      times.
    \end{question}

    Consider the following sequence of $n$ requests:
    $[0, 1, 2, \dots, m, 0, 1, 2, \dots, m, \dots]$.  Using LFU, there will be a
    total of $n$ page faults, whereas the best strategy would be to simply evict
    the most recently used page, which would result in $m$ page faults
    initially, plus one for every sequence of $0, 1, 2, \dots, m$ after.
  \end{problem}

\end{document}