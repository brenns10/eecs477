\documentclass[fleqn]{homework}

\student{Stephen Brennan (smb196)}
\course{}
\assignment{}
\duedate{}

%\usepackage{mathtools}
%\usepackage{graphicx}

\begin{document}
  \maketitle

  \begin{problem}{6}
    \begin{question}
      To provide adequate medical service to its constituents at a reasonable
      cost, hospital administrators must constantly seek ways to hold staff
      levels as low as possible while maintaining sufficient staffing to provide
      satisfactory levels of health care.  An urban hospital has three
      departments: the emergency room (department 1), the neonatal intensive
      care nursery (department 2), and the orthopedics (department 3). The
      hospital has three work shifts, each with different levels of necessary
      staffing for nurses. The hospital would like to identify the minimum number
      of nurses required to meet the following three constraints: 

      \begin{enumerate}[a.]
      \item The hospital must allocate at least 13, 32, and 22 nurses to the
        three departments over all shifts,
      \item The hospital must assign at least 26, 24, and 19 nurses to the three
        shifts over all departments, and
      \item The minimum and maximum number of nurses allocated to each
        department in a specific shift must satisfy the following limits:

        \begin{tabular}{|l|ccc|}
          \hline
          & Department 1 & Department 2 & Department 3 \\
          \hline
          Shift 1 & (6, 8) & (11, 12) & (7, 12) \\
          Shift 2 & (4, 6) & (11, 12) & (7, 12) \\
          Shift 3 & (2, 4) & (10, 12) & (5, 7) \\
          \hline
        \end{tabular}
      \end{enumerate}
    \end{question}
  \end{problem}

  \begin{problem}{8}
    \begin{question}
      Formulate the contractor problem (Homework Assignment 1, problem 7) as a
      minimum cost network flow problem, write its dual and complementary
      slackness conditions, and prove that this problem has always an integer
      optimal solution.
    \end{question}
  \end{problem}

\end{document}