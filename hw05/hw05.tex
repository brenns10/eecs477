\documentclass[fleqn]{homework}

\student{Stephen Brennan (smb196)}
\course{EECS 477}
\assignment{Homework 5}
\duedate{November 23, 2015}

\usepackage{mathtools}
\usepackage{graphicx}
\usepackage{enumerate}

\begin{document}
  \maketitle

  \begin{problem}{1}
    \begin{question}
      Apply the capacity-scaling algorithm to the minimum cost flow problem in
      the figure.

      \includegraphics[width=0.6\textwidth]{p1.png}
    \end{question}
  \end{problem}

  \begin{problem}{2}
    \begin{question}
      In this problem, you will revisit the gambling chip game.  Assume that
      there are four chips of value 6, 5, 2, 7, and thus the game ends in two
      rounds.

      \begin{enumerate}[a.]
      \item Write the game in extensive form, and then convert the extensive
        form into strategic form.
      \item Model the strategic form game as a linear program and find the
        optimal mixed strategies for the two players.
      \item Assume that Bob uses his optimal mixed strategy.  Describe a
        deterministic strategy that Alice can use to maximize her winnings.
      \item Find an arrangement of at least two rounds for which the optimal
        strategy is deterministic for both players.
      \end{enumerate}
    \end{question}
  \end{problem}

  \begin{problem}{3}
    \begin{question}
      Consider a data structure implemented as a linked list containing $n$
      distinct items.  The data structure supports the operation
      \texttt{CONTAINS(item)}, which returns true if the list contains the given
      item and false otherwise.  A \texttt{CONTAINS(item)} takes time
      proportional to $O(p)$, where $p$ is the position of the item in the
      list.  Furthermore, once \texttt{CONTAINS(item)} has located the item, it
      can move it forward to any position between the first and the $p$th within
      the $O(p)$ time bound.  In this way, if the same item is being requested
      repeatedly, subsequent \texttt{CONTAINS(item)} will take less time.
      Consider an algorithm for determining whether to move the item forward and
      to which location.

      \begin{enumerate}[a.]
      \item Show that $p \ge n$ in the worst-case for any deterministic
        algorithm.
      \item Use Yao's principle to show that $p \ge (n+1) / 2$ in the worst case
        for any randomized algorithm.
      \end{enumerate}
    \end{question}
  \end{problem}

\end{document}